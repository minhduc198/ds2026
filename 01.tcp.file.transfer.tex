
\documentclass{article}
\usepackage{graphicx}
\usepackage{hyperref}
\usepackage{listings}
\usepackage{caption}
\usepackage{geometry}
\geometry{margin=1in}

\title{TCP File Transfer Protocol Report}
\author{Your Team}
\date{}

\begin{document}
\maketitle

\section{Protocol Design}
Our TCP file transfer protocol aims to provide reliable file delivery between a client and a server. The protocol defines message types, connection steps, error handling, retransmission behavior, and file integrity validation. The client initiates the transfer and the server responds with success or failure signals.

\section{System Organization}
The system consists of two components:
\begin{itemize}
    \item \textbf{Server}: Listens for incoming connections and receives files.
    \item \textbf{Client}: Connects to the server and uploads files.
\end{itemize}

\begin{figure}[h]
\centering
\includegraphics[width=0.7\textwidth]{system_architecture.png}
\caption{System Architecture Diagram}
\end{figure}

\section{Implementation Details}
Below is the full implementation of the server and client.

\subsection{Server Code}
\begin{lstlisting}[language=Python]
# server.py
import socket

HOST = "127.0.0.1"
PORT = 65432

server = socket.socket(socket.AF_INET, socket.SOCK_STREAM)
server.bind((HOST, PORT))
server.listen()

print("Server listening...")

conn, addr = server.accept()
print(f"Connected by {addr}")

filename = conn.recv(1024).decode()
conn.sendall(b"Filename received")

with open(filename, "wb") as f:
    while True:
        data = conn.recv(4096)
        if not data:
            break
        f.write(data)

print("File received successfully.")
conn.close()
server.close()
\end{lstlisting}

\subsection{Client Code}
\begin{lstlisting}[language=Python]
# client.py
import socket

HOST = "127.0.0.1"
PORT = 65432

client = socket.socket(socket.AF_INET, socket.SOCK_STREAM)
client.connect((HOST, PORT))

filename = "example.txt"
client.sendall(filename.encode())
client.recv(1024)

with open(filename, "rb") as f:
    client.sendall(f.read())

client.close()
\end{lstlisting}

\section{Who Does What}
\begin{itemize}
    \item Member A: Designed the protocol.
    \item Member B: Implemented the server.
    \item Member C: Implemented the client.
    \item Member D: Wrote documentation and testing.
\end{itemize}

\section{How to Run}
\begin{enumerate}
    \item Start the server: \texttt{python3 server.py}
    \item Run the client: \texttt{python3 client.py}
    \item Ensure both scripts are in the same directory.
\end{enumerate}

\end{document}
